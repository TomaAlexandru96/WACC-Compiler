\documentclass[11pt]{article}

\usepackage{fullpage}
\usepackage{enumitem}
\usepackage{graphicx,wrapfig}
\usepackage{listings}

\begin{document}

\title{WACC Final Report}
\author{
	Alexandru-Iosif Toma,
    Mihai Vanea,
	Arthur-Mihai Niculae,
    Fraser Price
}
\maketitle

\section{Introduction}
Building a compiler from scratch for a tiny procedural while-language was a really interesting task to achieve and this project occupied most of our time for one month and a half. Following is our final project report. In terms of implementations, we start by introducing an overall analysis of our compiler, discussing about the design decision we had to take, finishing with an evaluation of our extension. We conclude our report with a group reflection and also personal reflection of each teammate.

\section{Overview}
\iffalse
THE PRODUCT(1/4)
1. An analysis and critical evaluation of the quality of the WACC compiler
2. It meets the functional specification?
3. Judge that it forms a sound basis for future development
4. May wish to address performance issues.
\fi
Our product is a compiler for a tiny while-language called WACC. The functionality it provides includes lexical, syntax and semantic analysis and also code generation for a subset of the ARM11 architecture. It is fully functional and every part was properly tested. We also believe that our design is easy to extend given the way we structured the compiler so far and how much time we've spent thinking about the design.

\section{Project Management}

\iffalse
THE PROJECT MANAGEMENT(2/4):
1. Analysis of the organisation of your group
2. Use of project management tools (such as Git)
3. How your group was structured
4. How you coordinated your work and detail any tools that helped/hindered your progress
5. What went well and what you would do differently if you were to do the lab again
\fi

This projects didn't expose as many challenges in terms of coordinating as we expected before starting it. When we formed the group we already knew the main skills and knowledge of our teammates and tried to make the skill set as diverse as possible. Our communication patterns were very regular because we've learned from previous projects that a solid conversation binds a team together. We understood the talents of each other before we started to allocate individual tasks  and we always kept in mind how crucial it is to actively support each other when help was requested. Thus, the moment of frustrations were minimal and every time something didn't go as planned or expected, we always remained calm and approached the problem in a systematic and well defined way. In future, we would like to maintain our communication habits, the way we approach and think about solving given tasks, give each other constant feedback and always remember that we are all part of a team and it is not about the success of an individual, but all of us.

\subsection{Communication to be changed !!!!!}
As mentioned before, communication was for us one of the key aspects which made our project go smoothly in the right direction. Right from the beginning we established clear communication channels. One interesting point to make here is that every team member had the pleasure to behave like a leader throughout the project, which is from our point of view it had a huge impact. Even if you have a very good leader assigned for the group, it there exists a time where another person has a better idea about solving a problem, then he must know that is the time to step back and let him take control over the situation. We managed to synchronize our enthusiasm and focus, therefore almost every day one of us used to message the entire team when do we want to meet and work together on the current tasks, and all of us agreed on a convenient time and met in the labs to work.

\subsection{Work distribution}
Most of the time we worked together, all 4 of us, especially in the beginning when it was vital given the design decisions which were involved. We knew that spending a little bit more time thinking on the design of the projects and the way we are going to implement it will have an impact on how easy we can extend the compiler afterwards. Our main focus was to make our code easy to understand and also if somebody will decide to add extra functionality to the code, they won't have to change the existing one (risking to change the behavior and correctness). After we came up with the design and knew what we are going to do next, we finally managed to split up the work and we adopted the pair programming technique learned from Software Engineering Design course. When we weren't together and somebody had a good idea on how to solve a bug or functionality they just worked on it individually. We also made extensive use of a management tools like Git to make sure we are as productive and fast as possible. For every package we had in our project, we created a branch in which we've only dealt with it's functionality and after we properly tested our implementation we merged it to master so that other teammates can make use of it as well.

\section{Compiler Design}
\iffalse
THE DESIGN CHOICES(3/4)
1. An analysis of the design choices that you made during the WACC lab
2. Design patterns you used when designing your code and why you chose to use them.
\fi
The design was one of the most challenging parts of the project due to the fact that it can heavily influence how easy it is to extend your code, trying to add more functionality to it, but without having to change the existing code in order to make space for it. Throughout all the phases of the project we made extensive use of some design patterns which made our life easier and served the purpose of making our code clean, understandable and also extendable:
\begin{itemize}[noitemsep,topsep=0pt]
  \item \textbf{Visitor}:
    In order to do the semantic and syntax checks for Front-End we used the visitor pattern. Because we chose Java to code our compiler, we were able to use ANTLR4 library to perform the lexical analysis (split the input file into tokens) and create an abstract syntax tree. After getting the tree, we created a SemanticVisitor class which had a method for visiting each rule created by the parser, returning a node corresponding to the visited rule.
  \item \textbf{Exception}:
    When dealing with semantic checks for the Front-End we used the Exception pattern. Due to the fact that we didn't want our compiler to stop when it reached a semantic error, we made every node from our enriched AST hold a list of the compile time errors encountered so far during the traversal of the tree. Every node was responsible for checking if there exists a semantic error within it (E.g.: an IfNode just checks if the condition is a boolean, if not adds a semantic error to the list, prints it and continues the traversal).
  \item \textbf{Enriched AST}:
    Due to the fact that we wanted each node from the tree created by the parser to hold some specific information which we considered necessary in order to fulfill our tasks, we created an enriched AST and consequently a node for every rule which was part of the language(E.g.: ReturnNode, PairNode, AssignLhsNode, etc.). In this way we also made the code extendable, in the sense that, if somebody else wants to improve our compiler, they can freely add more needed information in these nodes and fulfill their functionality.
  \item \textbf{Abstract methods for instructions}:
    This was the point where our design decisions made so far really stood out and shined. When we got to the task of generating machine code for Back-End, it was fairly easy to add the new functionality. We created a CodeGenerator class which was responsible for creating the labels, messages and instructions for the program. Here our Front-End beautifully merged with the Back-End, in the sense that we just made every node from our enriched AST be responsible for generating it's own code and CodeGenerator just brought everything together after all the nodes finished their job. Hence, we didn't need to change anything implemented so far, we just added extra functionality.
\end{itemize}


\section{WACC Extension}
\iffalse
BEYOND THE SPECIFICATION(4/4)
1. An evaluation of your extensions to your WACC compiler
2. Describe all of the language extensions, optimisations or other aspects that you have added to your compiler, including how these features can be accessed or viewed
3. BRIEFLY discuss what future extensions you would like to add to your WACC compiler if you had more time
\fi

We have implemented two control flow structures:
\begin{itemize}
\item \textbf{For loop} - This structure represents the C-style for loop with the following syntax:
\begin{lstlisting}
for (<assignemnt>; <boolean condition>; <incremental step>) do
	<statements>
done
\end{lstlisting}
The first step of the for control flow is the assignment of a loop variant which will be modified at each loop iteration according to the instruction from the incremental step. The loop then executes the statements between the scope block while the boolean condition holds. After the for loop is exited all the variables declared within the scope as well as the loop variant are destroyed.
\item \textbf{Do-while loop} - This structure represents the C-style do-while loop and has the following syntax:
\begin{lstlisting}
do
	<statements>
while <boolean condition> done
\end{lstlisting}
This control flow structure is almost identical to the while loop, the only difference being that the body of the do-while loop executes at least once regardless of the condition.
\end{itemize}

We have heavily tested our extension by implementing a ruby test suite. We have tested the syntax and semantic part of the extension as well as the code generation correctness. When we started the extension we were initially thinking of implementing a garbage collector for our language but given the short period of time, we found that the task was quite challenging and not so easy to accomplish given the time constraints.

\section{Acknowledgement}
We gratefully acknowledge the person who wrote the ANTLR4 library (Terence Parr) which helped us when working on the Front-End.

\end{document}
